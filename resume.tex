\documentclass[margin,line]{res}
\usepackage{url}
\usepackage{hyperref}

\oddsidemargin -.5in
\evensidemargin -.5in
\textwidth=6.0in
\itemsep=0in
\parsep=0in
% if using pdflatex:
%\setlength{\pdfpagewidth}{\paperwidth}
%\setlength{\pdfpageheight}{\paperheight} 

\newenvironment{list1}{
  \begin{list}{\ding{113}}{%
      \setlength{\itemsep}{0in}
      \setlength{\parsep}{0in} \setlength{\parskip}{0in}
      \setlength{\topsep}{0in} \setlength{\partopsep}{0in} 
      \setlength{\leftmargin}{0.17in}}}{\end{list}}
\newenvironment{list2}{
  \begin{list}{$\bullet$}{%
      \setlength{\itemsep}{0in}
      \setlength{\parsep}{0in} \setlength{\parskip}{0in}
      \setlength{\topsep}{0in} \setlength{\partopsep}{0in} 
      \setlength{\leftmargin}{0.2in}}}{\end{list}}

\begin{document}

\name{Krishna Vaidyanathan \vspace*{.1in}}

\begin{resume}
\section{\sc Contact Information}
\vspace{.05in}
\begin{tabular}{l l}
DC 2569				    & {\it Mobile:} +1-226-978-2760\\ 
Cheriton School of Computer Science  & {\it E-mail:}  \href{mailto:kvaidyan@uwaterloo.ca}{kvaidyan@uwaterloo.ca}\\    
University of Waterloo	             & \\
200 University Avenue West           & \\
Waterloo, Ontario, Canada	         & \\
N2L 3G1     
\end{tabular}

\section{\sc Current Position}
First year student of the M.Math program at the Cherition School of Computer Science, University
of Waterloo.

\section{\sc Academic Interests}
Reconfiguration problems, parametrized algorithms, and graph coloring. 
\section{\sc Education}
\begin{tabular}{l r}
\textit{University of Waterloo, Canada} & September, 2015 - May, 2017 (expected)\\
  \textbf{(M.Math Computer Science)} & \textbf{(88.75/100)}\\ 
\textit{PSG College of Technology, Coimbatore} & June, 2010 - May, 2015 \\
\textbf{(M.Sc. Theoretical Computer Science)} & \textbf{(8.62/10)}\\
\end{tabular}

\section{\sc Industry Experience}
\begin{list2}
\item \textbf{Jan - July, 2015:} SDE Intern, \textit{Amazon Development Center, Chennai}, India.  Worked on
an internal tool to static analyze codebases by generating a graph of dependencies and isolates
sections of the code that is affected by check-ins.  Worked extensively on Facebook's pfff adding features relevant to the project, which was used to static code analyze.  Also ported the graph from facebook/pfff to TitanDB, and wrote queries to derive insights from the graph. \\Code: \url{https://github.com/krishnavaidy/pfff}
\end{list2}

\section{\sc Research Experience}
\begin{list2}
\item \textbf{Janaury - July, 2016:} Extended a course project on Opinion
  Dynamics in Agent Research to a paper collaborating with \textbf{Prof. Robin Cohen}.
  
\item \textbf{Sep, 2015 - present:} Research Assistant, \textit{University of
    Waterloo, Canada}. Working on reconfiguration problems in graph coloring with \textbf{Prof. Naomi Nishimura}.

\item \textbf{May - July, 2014:} Summer Intern, \textit{Indian Statistical Institute Chennai}, India. Worked under the guidance of \textbf{Dr. Mathew C. Francis}. Worked on a few problems in Contact graphs of L-shapes in the plane and $B_{k}$-VPG graphs.

\item \textbf{May -  November, 2013:} Research Intern, \textit{Indian Institute of Science, Bangalore}, India. Worked under the guidance of \textbf{Prof. L. Sunil Chandran}. Investigated rainbow matchings in the class of strongly edge-colored graphs and found a bound on the maximum rainbow matchings in terms of its minimum degree.

\item \textbf{Jan, 2013 - April, 2014:} Worked on the problem of counting triangulations in non convex polygons with \textbf{Prof. R.S. Lekshmi} (\textit{PSG College of Technology}).

\item \textbf{May - June, 2012:} Summer Intern, \textit{Institute of Mathematical Sciences, Chennai}, India. During this period, attended lectures and programming classes on various important topics of Theoretical Computer Science.
\end{list2}

\section{\sc Publications}
\begin{list2}
\item Jasine  Babu,  L.  Sunil  Chandran, Krishna  Vaidyanathan. \textit{``Rainbow matchings in strongly edge-colored graphs."} Discrete Mathematics 338.7 (2015):  1191-1196.
\end{list2}

\section{\sc Workshops attended}
\begin{list2}
\item \textbf{March $\mathbf{3^{rd}}$ - $\mathbf{8^{th}}$, 2014:} Attended the \textbf{``Advanced School on Parametrized Algorithms and Kernelizations"} (ASPAK), a one week intensive school on parametrized algorithms and kernelization at the \textit{Institute of Mathemtatical Sciences (IMSc), Chennai}.

\item  \textbf{May $\mathbf{21^{st}}$ - $\mathbf{31^{st}}$, 2012:} Attended the workshop \textbf{``Network Optimization and Security"} conducted by \textit{IMSc, Chennai}.
\end{list2}

\section{\sc Computer Proficiency}
\begin{tabular}{l l l}

\textbf{Languages}&: & OCaml, Python.\\
\textbf{Backend}&:	& MySQL.\\
\textbf{Platform}&:  & Linux, Windows.\\
\textbf{Tools}&:     &MATLAB, LaTex.\\
\end{tabular}

\section{\sc Extracurricular Activites}
\begin{list2}
\item  Completed levels N5 and N4 in the ``Japanese Language Proficiency Test" and pursuing level N3.
\item Active Toastmaster at the \textit{Coimbatore Toastmasters Club}.
\end{list2}

\section{\sc Links}
\begin{list2}
\item Github: \url{https://github.com/krishnavaidy}
\item LinkedIn: \url{https://www.linkedin.com/in/krishna-vaidyanathan-07663636}
\end{list2}

\end{resume}
\end{document}
